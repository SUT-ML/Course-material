یکی از اساتید دانشگاه از دانشجویانش ناراضی است و قصد دارد نمرات آنها را به نحوی ثبت کند که میانگین کلاس کمترین عدد ممکن و تعداد افراد پاس‌نشده بیشینه باشد. از طرفی، استاد نمی‌خواهد که بقیه متوجه قصد او شوند. برای این منظور، او تصمیم می‌گیرد تا میانگین را تا جای ممکن بالا نگه دارد که واحد آموزش به او شک نکند، و تعداد افراد پاس‌نشده نیز مشکوک نباشد. مقدار
\LR{p-value}
برای شک کردن واحد آموزش را 1 درصد و به صورت دو طرفه\footnote{\lr{two-way}} در نظر بگیرید.
\begin{itemize}
    \item فرض کنید کلاس این استاد 100 دانشجو دارد. اگر در حالت عادی ۵ درصد از دانشجویان درس را پاس‌ نشوند، با استفاده از تست فیشر بگویید که استاد می‌تواند حداکثر چند نفر را پاس نکند به گونه‌ای که واحد آموزش نیز به او شک نکند؟
\item فرض کنید پس از حذف افراد پاس‌نشده، میانگین درس 
$12.4$
است. اگر میانگین درس این استاد در ترم‌های گذشته 
$17.9$
بوده باشد و بدانیم نمرات دانشجویان از توزیع نامشخصی با انحراف معیار 4 پیروی می کند، با استفاده از 
\LR{z-test}
مشخص کنید این استاد می‌باست حداقل چند نمره نمودار خطی ساده (اضافه کردن یک مقدار ثابت به همه نمرات) روی نمره‌ها اعمال کند تا نمرات کلاس توسط آموزش مورد بررسی قرار نگیرد؟
\item شما از این استاد و واحد آموزش باهوش ترید! شما مشاهده می‌کنید که استاد از عمد هر دو 
\LR{p-value}
را به حد پایین چسبانده و مقدار آن‌ها تنها کمی بالاتر از 
\LR{p-value}
مرزی است و از آنجایی که شما یک دانشجوی باهوش هستید، می‌خواهید ثابت کنید استاد از عمد هر دو 
\LR{p-value}
را تنها کمی بالاتر از 
\LR{p-value}
مرزی قرار داده است.
بگویید چطور می‌توانیم مدارکی به آموزش ارائه دهیم که 
\LR{p-value}
مد نظر آنها را رد کرده و نشان دهد که قصد استاد چه بوده.(راهنمایی: از مستقل بودن تست‌ها در بخش اول و دوم استفاده کنید)
\end{itemize}